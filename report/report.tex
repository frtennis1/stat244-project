%        File: proposal.tex Created: Wed Nov 07 07:00 PM 2018 E Last Change: Wed
%     Nov 07 07:00 PM 2018 E
%
\documentclass[letterpaper, 12pt]{article}
\usepackage{bbm}
% \usepackage{hyperref
\usepackage[thm]{macros}
% \usepackage{natbib}
% \usepackage{nopageno}
\newgeometry{margin=1in}

\pagestyle{fancy}
\lhead{}
\chead{\textsc{High-dimensional GLMs}}
\rhead{}

\newcommand{\by}{\bm y}
\newcommand{\bx}{\bm x}
\newcommand{\bX}{\bm X}
\newcommand{\bW}{\bm W}
\newcommand{\stb}{\bm e}
\newcommand{\bbeta}{\bm \beta}
\newcommand{\one}{\mathbbm{1}}
\newcommand{\trans}{\intercal}
\newcommand{\bg}{\bm g}
\newcommand{\bgamma}{\bm \gamma}
\newcommand{\linearlasso}{\hat \bbeta_{\text{lasso}}}
\newcommand{\linearlassofitted}{\hat \by_{\text{lasso}}}
\newcommand{\empave}{\mathbf{P}_n}
\newcommand{\theoave}{\mathbf{P}}
\newcommand{\bZ}{\bm Z}
\newcommand{\err}{\mathcal{E}}

\begin{document}

\title{\textbf{Model Selection and Estimation in High-dimensional Generalized Linear Models}}
\author{\textbf{Francisco Rivera} \\ Harvard College \and \textbf{Jiafeng (Kevin) Chen} \\ Harvard College}
\date{\today}

\maketitle
\doublespacing
\section{Introduction}
\label{sec:intro}
The workhorse for high-dimensional regressions is the $\ell_1$ lasso penalty
\citep{tibshirani1996regression}. The original lasso is developed for fitting
(normal) linear models with the objective\footnote{We consider $\by$ an
$n$-vector of response variables whose $i$th element is $\by_i$. We consider the
covariate matrix $\bX$ ($n \times p$). In most settings, $\bX$ include an
intercept column, $\bx_{0} = \bm 1$, that is not regularized.} \begin{equation}
    \hat \bbeta_{\text{lasso}} = \argmin_{\bbeta}\, \frac{1}{2n}\norm{\by -
    \bX\bbeta}^2 + \lambda \norm{\bbeta}_1 - \lambda |\beta_0|, \, \lambda > 0.
    \label{eq:linear_lasso}
\end{equation}
The first paper about lasso in a GLM setting is \cite{park2007l1}: Consider a scalar GLM with likelihood \[
L(y; \theta,\phi) = \exp\br{\frac{y\theta - b(\theta)}{a(\phi)} + c(y, \phi)},
\]
where $g(\E[y]) = \eta = \bx^\trans \bbeta$ for some scalar function $g$. Thus
we may consider a direct extension of \eqref{eq:linear_lasso}: \begin{equation}
    \hat \bbeta_{\text{GLM-lasso}} = \argmin_{\bbeta}\,
    \underbrace{-\frac{1}{n}\sum_{i=1}^n \bk{y_i \theta(\bbeta)_i -
    b\pr{\theta(\bbeta)_i}}}_{\ell_n(\bbeta)} + \lambda \norm{\bbeta}_1 -
    \lambda |\beta_0|,
    \label{eq:glm_lasso}
\end{equation}
where \eqref{eq:glm_lasso} reduces to \eqref{eq:linear_lasso} if the likelihood
is Gaussian and $g$ is the canonical link for the Gaussian model, which is the
identity function. Note that the objective \eqref{eq:glm_lasso} is convex if we
use the canonical link, since the exponential family log-likelihood is concave
in $\theta$, and $\theta = \bx^\trans \bbeta$ is linear in $\bbeta$ for the
canonical link.

\section{Model Selection with Lasso}

Brute-force model selection scales poorly with respect to the number of
predictors. That is, when building a model to predict $\bm{y}$ from a subset of
the columns of $\bX \in \mathbb{R}^{n,p}$ (since not all columns may be
predictive), there are $2^p$ subsets of columns to regress on. Checking all of
these models against each other incurs a cost exponential in $p$ and becomes
prohibitively expensive in a high-dimensional setting. In class, we tackled this problem with forward-selection and backward-deletion,
a greedy algorithm that adds and removes covariates until a local optimal is
reached. 

We introduced the lasso \eqref{eq:linear_lasso} in \Cref{sec:intro}. We first
summarize the literature on lasso with OLS in
this section. Assume that $\by = \bX \bbeta^\star + \bm \epsilon,\, \bm\epsilon
\sim
\Norm(0, \sigma^2 \bm I_n)$. In this case, where possibly $p > n$, the OLS
estimate $\hat \bbeta
\in \argmin_{\bbeta} \norm{\by - \bX \bbeta}^2$ may not be unique, but all such
solutions have the same fitted values $\hat \by_{\text{OLS}} = \bX \hat\bbeta$.
\begin{prop}
    For some constant $C$, with $r = \text{rank}(\bX)$, then $
    \E\bk{
    \norm{X\bbeta^\star - \hat \by_{\text{OLS}}}_2^2 
    } \le Cr\sigma^2.
    $
\label{thm:ols_bound}
\end{prop} 
\begin{proof}
    Observe that \begin{align*}
\norm{\by - \hat \by}^2  = \norm{\bX \bbeta^\star - \bX\bbeta} + 
\norm{\bm \epsilon}_2^2 + 2 \bm \epsilon^\trans \pr{\bX \bbeta^\star -
\bX\bbeta},
\end{align*}
for any $\bbeta$. Note that $\norm{\by - \hat \by_{\text{OLS}}}^2 \le \norm{\bm
\epsilon}_2^2$. Thus \begin{align*}
\norm{\by - \hat \by_{\text{OLS}}}^2  &= \norm{\bX \bbeta^\star - \hat \by_{\text{OLS}}} + 
\norm{\bm \epsilon}_2^2 + 2 \bm \epsilon^\trans \pr{\bX \bbeta^\star - \hat \by_{\text{OLS}}} \le \norm{\bm
\epsilon}_2^2\\
\implies \norm{\bX \bbeta^\star - \hat \by_{\text{OLS}}} &\le
2\bm\epsilon^\trans\underbrace{\pr{\hat \by_{\text{OLS}} - \bX \bbeta^\star}}_
{\in C(\bX)}.
\end{align*}
Thus we can write $\hat \by_{\text{OLS}} - \bX \bbeta^\star = \bm \Phi \bm \nu$,
for some $n\times \text{rank}(\bX) =: n\times r$ matrix $\bm \Phi$ of
orthonormal basis of the columns of $\bX$. Thus \[
\frac{\bm\epsilon^\trans{\pr{\hat \by_{\text{OLS}} - \bX \bbeta^\star}}}{
\norm{\bX \bbeta^\star - \hat \by_{\text{OLS}}}} = \frac{\bm \epsilon^\trans
\bm \Phi \bm \nu}{ \norm{\bm \Phi \bm \nu}_2} = \pr{\bm\Phi^\trans\bm
\epsilon}^\trans \frac{\bm \nu}{ \norm{\bm\nu}} \le \sup_{\bm u : \norm{\bm u}
\le 1}  \pr{\bm\Phi^\trans\bm
\epsilon}^\trans u.
\]
Using the above inequalities, we show that \[
\norm{\bX\bbeta^\star - \hat \by_{\text{OLS}}}^2 \le 4 \pr{\sup_{\bm u, 
\norm{\bm u} \le 1}
\underbrace{(\bm\Phi^\trans \bm\epsilon )^\trans}_{\Norm(0, \sigma^2 \bm I_r)}
\bm u}^2 \implies \E\bk{\norm{\bX\bbeta^\star - \hat \by_{\text{OLS}}}^2} \le
4\sigma^2 r,
\]
by properties of the normal distribution.\end{proof}

Let $\linearlasso$ be the linear lasso estimate as in \eqref{eq:linear_lasso}.
Let $\linearlassofitted$ be its fitted value. We can show that a bound similar
to \Cref{thm:ols_bound} for the lasso, with $O_\P(\sqrt{\log (p/n)})$ replacing
the $O
(r/n)$ dependence (This is the so-called \emph{slow rate} for the lasso, see
\cite{tibshirani2015sparsity}). Using the fact that $\linearlasso$ must achieve
a better value than $\bbeta^\star$ for the objective in 
\eqref{eq:linear_lasso}, and an expansion similar to the one in the proof of
\Cref{thm:ols_bound}, we can show the inequality \begin{align*}
\frac{1}{2n} \norm{\bX \bbeta^* - \linearlassofitted}_2^2 &\le \frac{1}
{n}\underbrace{\bm
\epsilon^\trans \pr{\linearlassofitted - \bX \bbeta^*}}_{\le \norm{\bX^T
\bm\epsilon}_\infty \norm{\linearlasso - \bbeta^\star}_1 \text{(H\"older)}} +
\lambda \norm{\bbeta^
\star}_1 - \lambda \norm{\linearlasso}_1 \tag{\emph{Basic inequality} for the
lasso}.
\label{eq:basic} \\
&\le \lambda \pr{\norm{\linearlasso - \bbeta^\star}_1 + \norm{\bbeta^
\star}_1 - \norm{\linearlasso}_1 } \tag{On events $\Omega = \br{n^{-1}
\norm{\bX^T
\bm\epsilon}_\infty \le \lambda }$} \\
&\le 2\lambda \norm{\bbeta^\star}_1.
\end{align*}
We can show that $
\P(\Omega) \ge 1 - p^{-C/2 + 1}
$ if $\lambda = C\sigma \sqrt{\log (p) / n}$ for $C > 2$, assuming standardized
columns in $\bX$ \citep[see][]{tibshirani2015sparsity}. This leads to the
following 

\begin{prop}[Slow rate for the lasso]
    Let $\lambda =  C\sigma \sqrt{\log (p) / n}$ for $C > 2$, then with
    probability at least $1 - p^{-C/2+1}$, \[
    \frac{1}{n}\norm{\bX \bbeta^* - \linearlassofitted}^2 \le 2 
    \norm{\bbeta^\star}_1 \sigma \sqrt{C \frac{\log p}{n}}.
    \]
\end{prop}
How far is the lasso from the optimum? Suppose $k < p$ of $\bbeta^\star$ are
nonzero, then if we knew these active predictors (i.e. have an \emph{oracle}),
the
{oracle} OLS estimator achieves an error of
$k\sigma^2/n$. One can show that the error cannot get better than $\log(p)
k\sigma^2/n$ without the oracle, and the best-subset estimator---replacing
$\ell_1$-norm in \eqref{eq:linear_lasso} with the nonconvex $\ell_0$-norm, $\norm{\bbeta}_0 =
\sum_i
\one\pr{\bbeta_i \neq 0}$---achieves this optimum. Under certain conditions, we
can show that the lasso achieves a \emph{fast rate} of $\log(p) k\sigma^2/n$,
which is the optimal rate. \cite{van2009conditions} provide a review of relevant
conditions for achieving the fast rate. These conditions mostly forbid high
correlations among the active columns in $\bX$.  Moreover, with even more
restrictions, one could show that there exists a
$\lambda$ in
\eqref{eq:linear_lasso} for which the corresponding $\linearlasso$ achieves
perfect \emph{support recovery} with high probability: \[
\sgn(\linearlasso) = \sgn(\bbeta^\star).
\]
We defer the more technical discussions to \cite{buhlmann2011statistics}. 

Chapter 6 of \cite{buhlmann2011statistics} extends much of the linear results to
convex loss functions, covering scalar-valued GLMs with canonical links. Let
$\rho_{f_{\bbeta}}$ be a loss function with respect to prediction function
$f_\beta$ that maps data $\bm Z_i = (\bx_i, \by_i)$ to $\R$. Denote $\empave
\rho = n^{-1}\sum_i\rho(\bZ_i)$ the empirical risk, $\theoave \rho = n^{-1} \E
\bk{\sum_i \rho
(\bZ_i)}$ the theoretical risk, and $\err(f) = \theoave\pr{\rho_f - \rho_{f_0}}$
the excess risk,
where $f_0$ is a
minimizer of $\theoave \rho$ in some class $\mathbf{F} \supset \mathcal F = 
\{f_{\bbeta}: \bbeta \in \R^p\}$. Let $f^0_{\text{GLM}} = \argmin_{\mathcal F}
\theoave \rho$ be the GLM approximation to $f_0$, which is assumed to have low
excess risk. In the GLM context, let $\bbeta^\star$ be the parameters such that
$f_{\bbeta^\star} = f_{\text{GLM}}^0$. 
The lasso in this context produces \[
\hat \bbeta = \argmin_{\bbeta} \br{\empave \rho_{f_{\bbeta}} + \lambda 
\norm{\bbeta}_1 }.
\]
Define the \emph{empirical process} to be the gap between theoretical and
empirical risks for a $\bbeta$: $\br{v_n (\bbeta) = (\empave - \theoave)
\rho_{f_{\bbeta}}: \bbeta \in \R^p}$. The \eqref{eq:basic} can be generalized to
\[
\err\pr{f_{\hat \bbeta}} + \lambda \norm{\hat \bbeta}_1 \le -\pr{v_n (\hat
\bbeta) - v_n (\bbeta^\star)} + \lambda \norm{\bbeta^\star}_1 + \err\pr{f_
{\bbeta^\star}}
\]
Using the above, we show that, with probability that converges to 1 as $\lambda
\to 0$, \[
\err(f_{\hat \bbeta}) + \lambda \lVert\hat \bbeta \rVert_1 \le 2 \bk{
    \err(f_{\bbeta^\star}) + 2\lambda \norm{\bbeta^\star}_1
}  
\]
as $\lambda \to 0$, this shows a sort of \emph{consistency} result. For
instance, if the true $f$ is a GLM, then $\err(f_{\hat \bbeta}) \to 0 = 2
\err(f_{\bbeta^\star})$.  We can also show \cite[][Chapter 6.7]
{buhlmann2011statistics} an \emph{oracle inequality} similar to that in the fast
rate of the lasso, where under some restrictive regularity conditions, $\err(f_
{\hat \bbeta})$ shrinks at the oracle rate of $O\pr{k \log p / n}$. 

\section{Estimation}
\label{sec:estimation}

\subsection{Regularization path methods}
\cite{efron2004least} gives an efficient estimation procedure for the linear
lasso \eqref{eq:linear_lasso} called the Least Angle Regression (LAR), which
relies on the fact that the \emph{regularization path}---$\hat \bbeta_i$ as a
function of $\lambda$---is piecewise linear in \eqref{eq:linear_lasso}. Such a
structure is often unavailable in applications like \eqref{eq:glm_lasso}. \cite{park2007l1} mimicks the LAR procedure in GLM estimation. The high-level idea is to note the
non-zero coefficients for different penalizations $\lambda$ and constrain our
search to these models. We describe their procedure in this section.

In order to employ this method, we require a method to solve
\eqref{eq:glm_lasso}. For now, we suppose that we have such a method and can use
it as a black box. It is sufficient for us to know that this black box takes in
an initial $\bbeta^\text{init}$ and employs a descent-based method to converge
toward the optimal $\bbeta^*$. \cref{sec:estimation} follows-up by describing
the inner workings of different approaches.

The lasso penalty induces a sparsity in our optimal $\bbeta$. This sparsity
increases as our penalization term $\lambda$ becomes bigger. Indeed, if we take
$\lambda \to \infty$, then our solution is driven to $\bbeta \to \bm{0}$ because
the penalization term is all that matters, and the norm of $\bbeta$ is minimized
at $\bm{0}$. The entire domain when $\bbeta = \bm{0}$ is uninteresting, so we
initialize our algorithm at $\lambda_\text{max}$ which we define as the smallest
$\lambda$ such that there is only one non-zero coefficient.

Once we have initialize $\lambda$ in our algorithm to $\lambda_\text{max}$, we
will repeat three steps that \cite{park2007l1} outline,

\begin{enumerate}

\item Determine the step length by which to decrease $\lambda$. The intent is to
pick this such that the precisely one more coefficient becomes non-zero.
Introducing notation, we let $\lambda_k$ be the $\lambda$ value we consider in
the $k$th step, and we have to pick $\lambda_k - \lambda_{k+1}$.

\item Using a linear approximation, predict the value of $\bbeta$ for the
decremented $\lambda_{k+1}$.

\item Initialize our solver of \eqref{eq:glm_lasso} with the linear estimate as
$\bbeta^\text{init}$ and get the precise solution $\bbeta^*$ for the new
$\lambda$.

\end{enumerate}

doing so until either we have brought $\lambda$ all the way to 0, or we
determine that incremental variables are not improving the model. In order to
perform the second step, a linear approximation prescribes that,

\[ \bbeta_{k+1} \approx \bbeta_k + (\lambda_{k+1} - \lambda_k) \frac{\partial
\bbeta}{\partial \lambda}.\]

Denote by $\bX_A$ the matrix made up by the columns of $\bX$ whose corresponding
$\bbeta$ coefficient is non-zero, i.e. in the active set. Furthermore, let $\bW$
be the diagonal matrix with $i^\text{th}$ diagonal element given by,

\[ \frac{1}{\var(y_i)} \left( \frac{\partial \mu}{\partial \eta} \right)^2\]

which changes because the derivative is evaluated at a different location based
on the value of $\lambda$, so we define $\bW_k$ as the value of $\bW$
corresponding to $\lambda_k$. Then, we have that,

\[ \bbeta_{k+1} \approx \bbeta_k - (\lambda_{k+1} - \lambda_k) (\bW_A^\top \bW_k
\bX_A)^{-1} \sgn (\beta_k).\]

Note that in the above equation, we pretend $\bbeta_k$ and $\bbeta_{k+1}$ only
contain their non-zero components, and thus are not going to have all $p$
dimensions. This makes our algebra simpler and we can do this because the linear
approximation will not make a previously zero coefficient non-zero.

Finally, we just need to determine our step size. One simple choice could be to
use a constant step size. However, in different domains of $\lambda$, the same
unit change in $\lambda$ could have dramatically different implications for the
sparsity of the resulting $\bbeta$. If we were working with a lasso in linear
regression, $\bW$ would not change, and we could pick the step size precisely to
coincide with the point at which a coefficient of zero becomes non-zero.
\cite{park2007l1} suggest doing precisely this, with the caveat that because
$\bW$ does change, it will only be an approximation for the GLM case. 

\subsection{Coordinate descent methods}
Unfortunately, \cite{park2007l1}'s method often scales poorly with the size of the problem. The
most popular method---proposed by \cite{friedman2010regularization} and
implemented in \textsf{R}'s \textsf{glmnet} package---is \emph{cyclical
coordinate descent} with iteratively reweighted least squares. The idea is to
approximate $\ell_n(\bbeta)$ in \eqref{eq:glm_lasso} with a second-order Taylor
expansion, either globally for all parameters $\bbeta$ (for scalar-valued GLMs)
or locally with a single parameter $\bbeta_j$ (for vector-valued GLMs, such as
the multinomial logistic regression). Such an approximation yields a quadratic
function (in the scalar GLM case) \[
\ell_{Q}(\bbeta) = \frac{1}{n}\sum_{i=1}^n w_{i} (y_i - \bx_i^\trans \bbeta)^2.
\]
We then solve a local penalized least-squares problem:\begin{equation}
    \bbeta \gets \argmin_{\bbeta} \ell_{Q}(\bbeta) + \lambda \norm{\bbeta}_1.
    \label{eq:local_opt}
\end{equation}
via cyclical coordinate descent, i.e. by iteratively solving \begin{equation}
    \bbeta_j \gets \argmin_{\bbeta_j} \ell_{Q}(\bbeta) + \lambda \norm{\bbeta}_1,
    \label{eq:cyclic_opt}
\end{equation}
holding all other entries $\bbeta_{-j}$ fixed. \eqref{eq:cyclic_opt} has an analytical solution for the lasso penalty\footnote{\cite{friedman2010regularization} show a similar expression for the \emph{elastic net} penalty: \[
\lambda P_\alpha(\bbeta) = \lambda \pr{\alpha \norm{\bbeta}_1 + (1-\alpha) \norm{\bbeta}_2}.
\]}
\begin{equation}
    \bbeta_j \gets \frac{S\pr{\sum_{i=1}^N w_i x_{ij}\pr{y_i - \tilde y_i\uppr{j}}, \lambda}}{\sum_{i=1}^N w_i x_{ij}^2}, \, S(t, \gamma) = \sgn(t)\pr{|z|-\gamma}_+,\, \tilde y_i\uppr{j} = \bx_i^\trans \bbeta - x_{ij}\bbeta_j.
    \label{eq:ccd_update}
\end{equation}
We summarize the procedure described above in \Cref{alg:ccd}. 
\begin{algorithm}
\caption{Cyclic coordinate descent algorithm for solving \eqref{eq:glm_lasso} (scalar GLM case) in \cite{friedman2010regularization}}
\label{alg:ccd}
\begin{algorithmic}
\State{Initialize $\bbeta$}
\For{$\lambda$ on regularization path}
\While{$\bbeta$ has not converged}
\State{Approximate $\ell_n(\beta)$ by $\ell_Q(\beta)$}
\While{cyclical descent has not converged}
\For{$j$}
    \State{Update $\bbeta_j$ according to \eqref{eq:ccd_update}}
\EndFor
\EndWhile
\EndWhile
\State{$\hat{\bbeta}_\lambda \gets \bbeta$}
\State{Initialize $\bbeta$ for next iteration to $\hat{\bbeta}_\lambda$}
\EndFor
\end{algorithmic}
\end{algorithm}

The machine learning literature slightly alters \Cref{alg:ccd} and changes \eqref{eq:ccd_update} into \[
\bbeta_j \gets S\pr{\bbeta_j - \pr{\nabla_{\bbeta} \ell_n(\bbeta)}_j \kappa^{-1}, \frac{\lambda}{\kappa}}
\]
for some \emph{learning rate} $1/\kappa$, in keeping with gradient descent. Moreover, \cite{shalev2011stochastic} proves a convergence guarantee for stochastic coordinate descent in this fashion, where, instead of cycling through the coordinates of $\bbeta$, a coordinate is chosen uniformly at random. 
\begin{theorem}
Let $Q(\bbeta)$ be the objective in \eqref{eq:glm_lasso}. 
    At iteration $T$ of the first while-loop in a verison of \Cref{alg:ccd} with stochastic coordinate descent and gradient updates, \[
    \E[Q(\bbeta_T)] - \E[Q(\hat\bbeta_{\text{GLM-lasso}})] \le C\frac{p \kappa}{T+1}
    \]
    for constant $C$ a function of the initial starting value $\bbeta\uppr{0}$, assuming that $\ell_n$ is differentiable with \[
    \ell_n (\bbeta + \eta \stb_j) \le \ell_n(\bbeta) + \eta \pr{\nabla \ell_n}_j + \frac{\kappa}{2} \eta^2
    \]
    for all $\eta, \bbeta, j$.\footnote{This condition restricts the choice of $\kappa$ as a function of the loss criterion.}
\end{theorem}
\begin{cor}
    The runtime to achieve $\epsilon$ expected accuracy is bounded by \[O\pr{\frac{np\kappa }{\epsilon} \norm{\hat \bbeta_{\text{GLM-lasso}}}_2^2}.\]
\end{cor}

Moreover, \cite{bradley2011parallel} show that a parallel version of the
coordinate gradient descent procedure above where at each iteration, $\mathsf P$
(possibly duplicate) coordinates are updated in parallel. For correlated
features, such parallelism is dangerous, since updating two correlated features
simulataneously may over or undercompensate for the gradient direction.
\cite{bradley2011parallel} quantifies the interference due to correlated
features and shows that efficiency increases linearly in the number of parallel
processes $\mathsf{P}$ so long as $\mathsf{P} \le \frac{p}{\rho}$ where $\rho$
is the largest modulus of the eigenvalues of $X^\trans X$.

Coordinate descent methods described above can also become expensive if $n$ is
large. The standard machine learning and optimization answer to this problem is
to use \emph{stochastic gradient descent}, replacing $\nabla_{\bbeta}
\ell_n(\bbeta)$ with an unbiased estimate $\bm g_i = \nabla_{\bbeta} \log L(y_i;
\bbeta)$, which is the gradient evaluated on a single observation.\footnote{We
can replace this with \emph{batched gradient descent} as well, where the
gradient estimate is averaging over a batch of observations.}
\cite{shalev2011stochastic} consider a mirror descent algorithm in the lasso
context, by running stochastic gradient descent on the dual problem and
enforcing sparsity in an intelligent manner. Let $\bgamma = f(\bbeta)$ be the
dual parameter for $\bbeta$ with an invertible link $f$. We choose an
observation $i$ at random, compute $\bg_i$, and update \begin{align*}
\bgamma &\gets \bgamma - \eta \bg_i \\
\bgamma' &\gets \bgamma - \eta\lambda \sgn(\bgamma) \tag{Decrease $\norm{\bbeta}_1$}\\
\bgamma_j &\gets \bgamma'_j \one\pr{\sgn(\bgamma_j) = \sgn(\bgamma'_j)} \tag{Maintains sparsity} \\
\bbeta &\gets f^{-1}(\bgamma).
\end{align*}
The runtime bound for the stochastic mirror descent algorithm in
\cite{shalev2011stochastic} is \[O\pr{\frac{p\log p}{\epsilon^2} \norm{\hat
\bbeta_{\text{GLM-lasso}}}_2^2}.\] We pay the price of the $p\log p$ and
$\epsilon^{-2}$ dependence, as opposed to $p$ and $\epsilon^{-1}$, in order to
achieve the benefit of a $n$-free runtime.



\newpage
% \nocite{*}
\bibliographystyle{jpe}
\bibliography{sources}

\end{document}


